\chapter{Robot-\/entrep}
\hypertarget{md_README}{}\label{md_README}\index{Robot-\/entrep@{Robot-\/entrep}}
\label{md_README_autotoc_md0}%
\Hypertarget{md_README_autotoc_md0}%
 Les malheurs de l\textquotesingle{}entrepôt est un sujet proposé par Mr Saint-\/\+Bauzel.

Ce projet est constitué de différent fichier \+: makefile, déclaration, main, interface, déplacement du robot, commande du robot, calcul du score et SDL2.

Pour installer la bibliothèque de SDL2 (sur linux)\+: sudo apt-\/get install libsdl2-\/dev

Une fois l\textquotesingle{}installation de la bibliothèque il suffit de faire un make run, puis de sélectionner l\textquotesingle{}un des quatre modes d\textquotesingle{}expériences\+: taper \textquotesingle{}o\textquotesingle{} pour l\textquotesingle{}entrepôt sur le terminal avec le fichier commande, taper \textquotesingle{}u\textquotesingle{} pour l\textquotesingle{}entrepôt sur le terminal avec le déplacement du robot en "{}zqsd"{}, taper \textquotesingle{}y\textquotesingle{} pour l\textquotesingle{}entrepôt sur la fenêtre de graphique avec le fichier commande, taper \textquotesingle{}i\textquotesingle{} pour l\textquotesingle{}entrepôt sur la fenêtre de graphique avec le déplacement du robot en "{}zqsd"{} ou \textquotesingle{}p\textquotesingle{} pour sortir (la selection \textquotesingle{}p\textquotesingle{} permet de sortir de l\textquotesingle{}entrepôt sur le terminal avec le déplacement du robot en "{}zqsd"{}). 